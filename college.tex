\documentclass[11pt]{article}
\usepackage[margin=1.25in]{geometry}
\setlength\parindent{0pt}
\setlength\headheight{15pt}

\usepackage{bm}
\usepackage{array}
\usepackage{float}
\usepackage{amsmath}
\usepackage{caption}
\usepackage{enumitem}
\usepackage{fancyhdr}
\usepackage{tcolorbox}
\usepackage[hidelinks]{hyperref}

\newenvironment{example}{
\begin{tcolorbox}[title=Example, colback=blue!5!white, colframe=black!75!blue]
}{ \end{tcolorbox} }

\renewcommand{\it}[1]{\textit{{#1}}}
\renewcommand{\bf}[1]{\textbf{{#1}}}
\newcommand{\ib}[1]{\textit{\textbf{{#1}}}}
\newcommand{\ul}[1]{\underline{{#1}}}
\renewcommand{\th}{$^{\text{th}}$ }

\pagestyle{fancy}
\fancyhf{}
\fancyhf[HL]{\it{College Advice}}
\cfoot{\thepage}
% \fancyhf[HR]{\it{Warren Kim}}

\begin{document}
\tableofcontents
\newpage

\section{Overview}
\label{sec:overview}
In this document, I plan to give some insight into my college journey as well as
some general advice for making the most out of your college career.

\paragraph{Note:} A lot of the material will be biased towards Computer Science
(where applicable). Additionally, \bf{everything from this point
on is \it{my opinion}.}

\vspace{-1em}
\paragraph{Note:} This document is pretty long. For convenience, the topics are
ordered from most to least important. The blue Example boxes are less important
than the plaintext.

\subsection{Introduction}
I am a transfer student (from IVC and Saddleback College) attending UCLA,
majoring in Computer Science, and graduating in the Spring of 2024. I will be a
Master's student at UCLA starting the Fall of 2024 studying Computer Science.

\section{Choosing a College}

\subsection{Ranking (Name Brand)}
A college's ranking (usually) \ib{does not matter}. Unless you want to attend a
school for the brand name (which is fair), do more research into the field(s)
you are interested in!
\begin{example}
  The University of Illinois Urbana-Champaign (UIUC) is ranked 35\th overall
  but 5\th (above Cornell, Princeton, and UCLA!) in Computer Science.
\end{example}

While a college's rank gives a general heuristic for the quality their
education, it should not dictate where you go.


\subsection{Major Choice}
If you do not know what you want to study in college yet, that is \ib{okay}! I
\ib{do not} recommend applying for an ``easy to get in'' major with a plan to
switch once you are there. Some departments do not allow for students within the
school to change into a subset of majors.
\begin{example}
  It is (allegedly) extremely difficult to transfer into EECS at UC Berkeley
  from another major. At UCLA, if you are a transfer student and get admitted
  for a major that is outside the School of Engineering, they do not allow you
  change into it.
\end{example}

\subsection{Criteria}
Here are a few things you may want to consider before deciding where
to apply (or attend). The criteria for choosing a college differs from person to
person. Some questions to get you started are
\begin{itemize}[label=$\to$]
  \item What \it{exactly} do you want to study?
  \item What do you want to get out of college?
  \item Are you planning on going to graduate school?
\end{itemize}

We take a more in-depth look into these questions here.

\paragraph{What \ib{exactly} do you want to study?} This may be simple to
answer, but remember to consider \it{all} of your options. Even if you think you
know what you want to study, be sure to explore the nuances of your major!
\begin{example}
  Suppose you want to study Mathematics. The question then becomes ``What branch
  of Mathematics?''. You may want to pursue a Pure Mathematics degree if the
  answer is theory. If you want to research topics in industry, then you may
  want to major in Applied Mathematics. If you want to study the foundations of
  Mathematics, then maybe a Philosophy degree is more appropriate.
\end{example}

\paragraph{What do you want out of college?} This question is pretty
straightforward. ``I want to go to college to make money'' is a respectable
answer. So is ``I want to make friends''! As long as you are honest with
yourself, any answer is the correct answer.

\paragraph{Are you planning on going to graduate school?} Depending on your
major and career goals, graduate school may be an option or required. Even if
you aren't planning on going to graduate school, I recommend keeping it open as
an option. Your opinion might change as you go through college!
\begin{example}
  A graduate degree in Computer Science is usually not required for an entry
  level job. However, a graduate program in Computer Science typically dives
  more into the theory of computation as well as provides a structured
  environment to learn or specialize in particular branches of the discipline.
\end{example}


\newpage
\section{General Advice for Education}
\subsection{Courses and Course Load}
Take classes that are interesting to \it{you}. Explore outside of your major and
see what piques your interest (even if it doesn't \it{directly} help with your
major)!
\begin{example}
  One of my friends is a Cognitive Science major and is taking (hard) Computer
  Science electives because he was interested in the similarities of
  how humans and computers think and learn.
\end{example}

Take a managable course load! ``Managable'' means something different to
everybody, so use your best judgement.
\begin{example}
  Throughout community college and the first year and a half at UCLA, I was
  working an average of 72 hours a week and taking four or more major-related
  classes a semester/quarter. It was not the best for my sleep schedule and I
  wasn't able to focus as well in class. Once I quit my job, the grass got
  greener, the sky turned blue, and I was able to engage in the course material.
\end{example}

\subsection{Professors}
Talk to your professors! It sounds trivial but you would be surprised how empty
some office hours can be. Professors are experts in their field, and are a great
resource and mentors. Who knows, maybe they will offer you opportunities!
\begin{example}
  I got to know my Programming Languages professor very well, and we now climb
  together! I even presented an advanced topic in data structures as a guest
  lecturer for his introductory Data Structures class.
\end{example}

\begin{example}
  I got to know Professor Paul Eggert pretty well, who is a well-respected
  developer in the open source community and a distinguished professor. He has
  made significant contributions to the Linux kernel, particularly in GNU
  \texttt{coreutils}, helped develop Emacs (a text editor similar to Vim), and
  currently maintains the timezone database (tz) backed by ICANN. I learned so
  much from his (lower division) Software Construction. He is extremely
  knowledgeable about the Unix system and systems computing in general (which is
  one of my many interests).
\end{example}

\subsection{Grades}
While grades are important, it isn't as important as actually \it{knowing} the
material. This rings true especially for  major courses. I recommend
prioritizing learning the material over chasing high grades. Typically, if you
know the material, good grades will follow. Quoting my professor,
\begin{quote}
  ``If I didn't have to give out grades, I wouldn't. Only \it{you} know how much
  you have grown throughout this course. Your grade is merely a reflection of
  three 2 hour \it{slivers} of time during the exams; it does not capture the
  full learning process.''
  \vspace{-1em}
  \begin{flushright}
    -- Professor Alexander Sherstov
  \end{flushright}
\end{quote}

\subsection{Aside: My Academic Journey}
I was a very bad student up to and including high school. I rarely did homework
in a timely manner, often doing it the period before it was due. I can't
remember a time I did homework at home or studied for an exam. I didn't take any
AP or Honors classes until junior year, and even then, it was because all of my
friends were taking them. Out of high school, I was rejected from almost every
UC and CSU that I applied to, and so I was (in a way) ``forced'' to go to IVC
(and eventually Saddleback).

At community college, I met some brilliant instructors and professors, and I was
very fortunate study under them. The slower pace of CC gave me an opportunity to
mature and figure out what I wanted to study at university. I was able to
maintain a good GPA throughout CC and got into every UC and CSU I applied to at
the end of my second year. I ended up choosing to go to UCLA because it was
comparatively cheaper than Berkeley.
\vspace{1em}

This anecdote is here to (hopefully) alleviate some worry about getting into a
university right out of high school, or not knowing what you want to study. I
took a pretty nontrivial path to university, but it definitely worked out for
the better.
\begin{center}
  \vspace{5em}
  \bf{$\bm{\sim}$ The rest of this page is intentionally left blank. $\bm{\sim}$}
\end{center}

\newpage
\section{The UC System}
\label{sec:uc}
You were probably told that UC's are more research oriented and CSU's are more
applied. While that certainly is true, there are some things you should keep in
mind. Since I only have experience with UC's, I will focus the conversation
on them.

\subsection{Professors at a UC:}
Depending on the department, most professors at a UC are hired for research and
not for teaching. This is one of the reasons why you tend to learn topics in a
more theoretical context at UC's as opposed to CSU's.

\subsection{Coursework at a UC:}
To build atop of the last point, coursework is typically more theoretical. So,
even if you study an applied science, you may learn more theory than you expect.
\begin{example}
  Most upper-division Computer Science courses at UCLA are \it{very} theoretical.
  The amount of actual coding I do for coursework is minimal. I often write out
  homework with pen and paper, since most of the questions assigned are not
  programming tasks, but problem solving/proof-based questions. The
  coding I did do for assignments typically did not directly relate to the
  course material. However, \it{because} of my theory courses, I now better
  understand the (seemingly infinite) web of Computer Science. I am
  able to solve new and unique problems by applying, in part, the theory I
  learned as well as my general software engineering skills.
\end{example}

\bf{One \ib{very} important note:} You should \ib{not} expect to learn ``everything''
from your classes. In order to get the most out of the course, you should be
researching topics outside the scope of the course. With how quickly the
quarters pass by (unless you're at UC Berkeley), there simply isn't enough time
to teach everything about a course topic in just ten weeks. Therefore, things
\it{will} be left out.
\begin{center}
  \vspace{5em}
  \bf{$\bm{\sim}$ The rest of this page is intentionally left blank. $\bm{\sim}$}
\end{center}

\newpage
\section{Opinions}
This section details my \ib{personal outlook} on education.
\subsection{Theoretical Coursework}
I believe that the theory behind your field is the most important thing you can
learn. This is because I believe that it is important to know \it{why} you are
doing the things you are, not just \it{how}. With the sheer amount of
information on the internet, you are objectively able to learn \it{how}
something works, or \it{how} to do a job (\it{especially} now due to tools like
ChatGPT). However, a formal course provides a structured environment to learn
\it{why} these things work. They will most definitely \ib{not} be applicable to
day-to-day life, but the process of learning theory gives you an insight on how
to solve new, creative problems.
\begin{example}
  I took abstract math courses (Algebra, Number Theory, and Set Theory) that
  outline the foundations of the Mathematics we are familiar with. While they
  don't directly make me better at Math or Computer Science, it gave me very
  important insight into how I can solve new problems with the tools I am
  given. For example, Number Theory and Ring Theory are heavily used in the
  field of Cryptography. The RSA encryption algorithm is built using concepts
  from Ring Theory. Set Theory is heavily used in Database and Programming
  Language Theory, which gives me insight into the limits of computation in
  those disciplines.
\end{example}

\subsection{Purpose of College}
I believe the purpose of college is to immerse yourself into a field of study
that \ib{you enjoy}. You often hear people say ``I don't use anything I learned
from college at my job'', which may or may not be true. However, education and
industry typically have disjoint philosophies. Education typically requires you
to think rigorously about a question. Industry (typically) just wants things to
get done. You are given (roughly) four years to explore the topic(s) of your
choice, so you should take advantage of it!
\begin{center}
  \vspace{5em}
  \bf{$\bm{\sim}$ The rest of this page is intentionally left blank. $\bm{\sim}$}
\end{center}

\newpage
\subsection{Ancecdote: Professors}
\ib{In my opinion}, a truly captivating professor has the ability to change your
entire outlook on education, which was what happened to me. I originally
attended college just to get a job, and so I tried to rush through it by taking
as many courses per quarter as I could. However, my entire outlook changed after
taking \ib{one} course (Theory of Computation). It was the first time I had met
a professor so passionate about not only his research, but teaching as well. He
is the sole reason I decided to pursue graduate school.
\begin{example}
  Professor Alexander Sherstov is a brilliant professor, and has an unparalleled
  passion for teaching. His Theory of Computation course (traditionally one of the
  most challenging courses in any Computer Science curriculum) was made intuitive
  only because we were being taught by someone who had a true passion for
  teaching. His teaching philosophy has no doubt transformed generations of
  students.
  \begin{quote}
    ``I firmly believe that research and teaching are completely disjoint
    skills. A lot of times students, faculty, and administration believe that
    those who conduct important and compelling research can naturally teach
    subjects well, but this really is not true. A teacher must really keep the
    students' best interest and understanding in mind, but in many cases,
    professors do not necessarlily care about this since their first and
    foremost interest is their research. Even textbooks are often written from a
    researcher's perspective rather than a student's perspective; and again, the
    student loses.''
    \vspace{-1em}
    \begin{flushright}
      -- Professor Alexander Sherstov
    \end{flushright}
  \end{quote}
\end{example}

Towards the end of the quarter, he said something that really stuck with me, and
it changed my outlook on college as a whole.
\begin{quote}
  ``We started with nothing but first principles, and worked our way up to
  Turing machines! It has been an absolute joy to walk this journey with you all
  these past ten weeks. Now, you know the theory behind what we call
  \it{computation}, and no one can take that away from you. You should all be
  proud of yourselves; this course is the most difficult course in the Computer
  Science curriculum here at UCLA. No matter how much you \it{think} you have
  learned, the reality of it is that you have learned so
  much more.''
  \vspace{-1em}
  \begin{flushright}
    -- Professor Alexander Sherstov
  \end{flushright}
\end{quote}
\begin{center}
  \vspace{5em}
  \bf{$\bm{\sim}$ The rest of this page is intentionally left blank. $\bm{\sim}$}
\end{center}

\newpage

The personal anecdote he gave in his last lecture was also very wholesome.
\begin{quote}
  ``About twenty years ago I was in your shoes, and when I learned about
  undecidability for the first time, and it shook me to my core. For a few days,
  food didn't taste the same, I couldn't sleep well, and life really sucked. I
  thought \it{How can this be? I have a theorem that tells me that the halting
  problem is undecidable?} Surely, if someone gives me some computer program,
  I'm going to roll up my sleeves, dig in, and find a solution. It may take some
  time, but if I put my mind to it, I'm going to find out if the Turing machine
  halts on that program.
  \vspace{1em}

  It really shook me because there's this cognitive dissonance between what you
  \it{feel} should be true and what you are \it{told} is true. What makes things
  worse is that there's someone telling you that something is \it{impossible}.
  When someone tells you that you cannot do something, all of a sudden, for some
  weird reason, you \it{really} want to do it, even if you weren't interested
  before.
  \vspace{1em}

  I wanted to resolve this seeming contradiction, and eventually, I found a way.
  Here's the thing: the theorem is right, but I'm \it{also} right in my own way.
  The theorem says we cannot decide the halting problem on \it{all} inputs, and
  that's fine. I never claimed that I would live forever to solve infinitely many
  inputs.
  \vspace{1em}

  Humans can only solve finitely may inputs by definition; you get to pick
  which inputs you solve, but it's always a finite number. \it{That} is the
  distinction. The theorem doesn't say that humans cannot solve \it{instances} of
  undecidable problems, it just tells us that humans cannot solve \it{all} of the
  infinitely many instances. It's just a biological limitation, not a cognitive
  one.
  \vspace{1em}

  You know, sooner or later, innocence has to give way to experience. What we
  want is to solve an infinite language, right? We want to rake in an entire
  constellation. But if we thought about it really hard, we would realize that
  getting a hold of even a \it{single} star, solving a language on a \it{single}
  input, would be a lifetime achievement\footnote{He has solve \it{multiple}
  problems that were previously thought to be unsolvable!}. That would make us
  happier people; it certainly made me happier.
  \vspace{1em}

  As you all graduate and move on with your careers, I know the future of
  computing is in good hands. Thank you so much; it has been a pleasure having
  you in this class. I hope you can see the beauty of computation as I do, and I
  hope you learned something about computation.''
  \vspace{-1em}
  \begin{flushright}
    -- Professor Alexander Sherstov
  \end{flushright}
\end{quote}


\newpage
\section{Statistics}
For those interested, these were my high school and community college
statistics.
\subsection{High School}
\begin{table}[ht]
    \centering
    \caption*{\it{Grade Point Averages}}
    \vspace{-0.8em}
    \begin{tabular}{l|c|c}
        Type       & Weighted     & Non-weighted \\
        \hline
        Academic 9-12  & 4.1538   & 3.7179 \\
        Academic 10-12 & 4.3793   & 3.7931 \\
    \end{tabular}
    \vspace{-1em}
\end{table}

% freshman
\begin{table}[H]
    \centering
    \caption*{\it{Freshman Year}}
    \vspace{-0.8em}
    \begin{tabular}{l|l}
        Course          & Grade \\
        \hline
        English 1A      & A$-$  \\
        Latin 1A        & A$-$  \\
        Math IIA        & A   \\
        Biology A       & B   \\
        Cultl Gbl Age A & B   \\
    \end{tabular}
    \hspace{2em}
    \begin{tabular}{l|l}
        Course      & Grade \\
        \hline
        English 1B      & A   \\
        Latin 1B        & B$+$  \\
        Math IIA        & A   \\
        Biology B       & B$+$  \\
        Cultl Gbl Age B & B$+$  \\
    \end{tabular}
    \vspace{-1em}
\end{table}

% sophomore
\begin{table}[H]
    \centering
    \caption*{\it{Sophomore Year}}
    \vspace{-0.8em}
    \begin{tabular}{l|l}
        Course          & Grade \\
        \hline
        English 2A      & A   \\
        Latin 2A        & A   \\
        Math IIIA       & A$+$  \\
        Chemistry A     & A   \\
        World History A & A   \\
    \end{tabular}
    \hspace{2em}
    \begin{tabular}{l|l}
        Course          & Grade \\
        \hline
        English 2B      & A   \\
        Latin 2B        & A$-$  \\
        Math IIIB       & A$+$  \\
        Chemistry B     & A   \\
        World History B & A   \\
    \end{tabular}
    \vspace{-1em}
\end{table}

% junior
\begin{table}[H]
    \centering
    \caption*{\it{Junior Year}}
    \vspace{-0.8em}
    \begin{tabular}{l|l}
        Course           & Grade \\
        \hline
        H American Lit A & A$-$ \\
        H Latin 3A       & A$-$ \\
        H Precalculus A  & A$-$ \\
        AP Statistics A  & A  \\
        AP Physics 1A    & C  \\
        AP US History A  & B  \\
    \end{tabular}
    \hspace{2em}
    \begin{tabular}{l|l}
        Course       & Grade \\
        \hline
        H American Lit B & A   \\
        H Latin 3B       & A$-$  \\
        H Precalculus B  & A   \\
        AP Statistics B  & A   \\
        AP Physics 1B    & A$-$  \\
        AP US History B  & A$-$  \\
    \end{tabular}
    \vspace{-1em}
\end{table}

% senior
\begin{table}[H]
    \centering
    \caption*{\it{Senior Year}}
    \vspace{-0.8em}
    \begin{tabular}{l|l}
        Course       & Grade \\
        \hline
        AP Eng Lit A   & B   \\
        AP Latin A     & A$-$  \\
        AP Calc BC A   & B$-$  \\
        AP Com Sci A   & B$+$  \\
        AP Macro Econ  & A$-$  \\
        Beg Ceramics A & A$+$  \\
    \end{tabular}
    \hspace{2em}
    \begin{tabular}{l|l}
        Course       & Grade \\
        \hline
        AP Eng Lit B   & CR  \\
        AP Latin B     & CR  \\
        AP Calc BC B   & CR  \\
        AP Com Sci B   & CR  \\
        AP Macro Econ  & CR  \\
        Beg Ceramics A & CR  \\
    \end{tabular}
\end{table}
\newpage

\subsection{AP Exam Scores}
\begin{table}[H]
    \centering
    \vspace{-0.8em}
    \begin{tabular}{l|l}
        Exam    & Score \\
        \hline
        AP Physics 1              & 2 \\
        AP Statistics             & 4 \\
        AP Calculus BC            & 4 \\
        AP U.S. History           & 3 \\
        AP Lit. and Comp.         & 4 \\
        AP Macroeconomics         & 5 \\
        AP Computer Science A     & 4 \\
        AP Environmental Science  & 3 \\
        AP U.S. Gov. and Politics & 3 \\
    \end{tabular}
\end{table}

\subsection{Community College}
\begin{table}[H]
    \centering
    \caption*{\it{Grade Point Average}}
    \vspace{-0.8em}
    \begin{tabular}{l|c}
        Type     & GPA  \\
        \hline
        Cumulative & 4.00 \\
        Department & 4.00 \\
    \end{tabular}
    \vspace{-1em}
\end{table}
% first year
\begin{table}[H]
    \centering
    \caption*{\it{First Year}}
    \vspace{-0.8em}
    \begin{tabular}{l|l}
        Course                               & Grade \\
        \hline
        C Programming                        & A \\
        H Psychology 1                       & A \\
        College Writing 2                    & A \\
        Analytical Geometry/Calculus III     & A \\
    \end{tabular}
    \hfill
    \begin{tabular}{l|l}
        Course                           & Grade \\
        \hline
        H Communications                 & A \\
        Java Programming                 & A \\
        Physics I (Kinematics)           & A \\
        Intro. to Linear Algebra         & A \\
        Intro. to Computer Systems       & A \\
    \end{tabular}
    \vspace{-1em}
\end{table}

% summer session
\begin{table}[H]
    \centering
    \vspace{-0.8em}
    \caption*{\it{Summer Session}}
    \begin{tabular}{l|l}
        Course                                 & Grade \\
        \hline
        Intro. to Computer Science I           & A \\
        H Principles of Microeconomics         & A \\
        Physics II (Electricity \& Magnetism)  & A \\
    \end{tabular}
    \vspace{-1em}
\end{table}

% second year
\begin{table}[H]
    \centering
    \vspace{-0.8em}
    \caption*{\it{Second Year}}
    \begin{tabular}{l|l}
        Course                             & Grade \\
        \hline
        Assembly Language I                & A \\
        H Film \& US Culture               & A \\
        Discrete Mathematics I             & A \\
        Intro. to Computer Science II      & A \\
        Elementary Differential Equations  & A \\


    \end{tabular}
    \hfill
    \begin{tabular}{l|l}
        Course                         & Grade \\
        \hline
        Physics III (General)          & A \\
        Assembly Language II           & A \\
        Discrete Mathematics II        & A \\
        H Academic, Career, Life       & A \\
        Intro. to Computer Science III & A \\
        Data Structures and Algorithms & A \\
    \end{tabular}
\end{table}
% PAGE BREAK COMMENT
% \begin{center}
%   \vspace{5em}
%   \bf{The rest of this page is intentionally left blank.}
% \end{center}
%
% \newpage
\end{document}
