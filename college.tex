\documentclass[12pt]{article}
\usepackage[margin=1in]{geometry}
\setlength\parindent{0pt}
\setlength\headheight{15pt}

\usepackage{array}
\usepackage{float}
\usepackage{amsmath}
\usepackage{caption}
\usepackage{enumitem}
\usepackage{fancyhdr}
\usepackage{hyperref}
\usepackage{tcolorbox}

\newenvironment{example}{
\begin{tcolorbox}[title=Example, colback=blue!5!white, colframe=black!75!blue]
}{ \end{tcolorbox} }

\renewcommand{\it}[1]{\textit{{#1}}}
\renewcommand{\bf}[1]{\textbf{{#1}}}
\newcommand{\ib}[1]{\textit{\textbf{{#1}}}}
\newcommand{\ul}[1]{\underline{{#1}}}
\renewcommand{\th}{$^{\text{th}}$ }

\pagestyle{fancy}
\fancyhf{}
\fancyhf[HL]{\it{College Advice}}
\fancyhf[HR]{\it{Warren Kim}}

\begin{document}
\tableofcontents
\newpage

\section{Overview}
\label{sec:overview}
In this document, I plan to give some insight into my college journey as well as
some advice for making the most out of your college career. The material is
intentionally terse since rigorous advice for college is typically handled on a
case-by-case basis.

\paragraph{Note:} A lot of the material will be biased towards Computer Science
and Mathematics (where applicable).

\subsection{Introduction}
I am a transfer student (from IVC and Saddleback College) attending UCLA,
majoring in Computer Science, and graduating in the Spring of 2024.

\section{College Overview}
This section covers two major aspects of colleges: ranking and major choice. It
aims to offer an alternate perspective on these two topics.

\subsection{Ranking (Name Brand)}
A college's ranking \ib{does not matter}. Unless you want to attend a school for
the brand name (which is fair), do more research into the field(s) you are
interested in!
\begin{example}
    The University of Illinois Urbana-Champaign (UIUC) is ranked 35\th overall
    but 5\th (above Cornell and Princeton!) in Computer Science.
\end{example}

While a college's rank gives a general heuristic for the quality their
education, it should not dictate where you go.


\subsection{Major Choice}
If you do not know what you want to study in college yet, that is \ib{okay}! I
\ib{do not} recommend applying for an ``easy to get in'' major with a plan to
switch once you are (hopefully) there. Some departments do not allow for
students within the school to change into a subset of majors.
\begin{example}
    It is (allegedly) extremely difficult to transfer into EECS at UC Berkeley
    from another major.
\end{example}

\newpage
\section{Choosing a College}
Choosing a college to apply for (or attend) is an important and difficult
decision. Here are a few things you may want to consider before deciding where
to apply (or attend).

\subsection{Criteria}
The criteria for choosing a college differs from person to person. Some
questions to get you started are
\begin{itemize}[label=$\to$]
    \item What major do you want to study?
    \item What do you want to get out of college?
    \item Are you planning on going to graduate school?
    \item Who would you like to study under?
\end{itemize}

We take a more in-depth look into these questions here.

\paragraph{What major do you want to study?} This may be simple to answer, but
remember to consider \it{all} of your options.
\begin{example}
    Suppose you want to do research in Mathematics. The question then becomes
    ``What branch of Mathematics?''. You may want to pursue a Pure Mathematics
    degree if the answer is theory. If you want to research topics in industry,
    then you may want to major in Applied Mathematics. If you want to study the
    foundations of Mathematics, then maybe a Philosophy degree is more appropriate.
\end{example}

\paragraph{What do you want out of college?} This question is pretty straightforward.
``I want to go to college to make money'' is a respectable answer.

\paragraph{Are you planning on going to graduate school?} Depending on your
major and career goals, graduate school may be an option or required. If you
want to go to graduate school, be sure you know \it{why} you want to go. ``I've
been in school for (about) 12 years, what's another 2-6?'' is a perfectly valid answer!
\begin{example}
    A graduate degree in Computer Science is usually not required for an entry
    level job. However, a graduate program in Computer Science typically dives
    more into the theory of computation as well as provides a structured
    environment to learn or specialize in particular branches of the discipline.
\end{example}

\paragraph{Who would you like to study under?} This
is the most important question. Remember, an education is only as good as the
educator!
\begin{example}
    For Computer Science, I wanted to study under Professor Paul Eggert because
    he has made significant contributions to the Linux kernel, particularly in
    GNU \texttt{coreutils}, helped develop Emacs (a text editor similar to
    Vim), and currently maintains the timezone database (tz) backed by ICANN.
    I learned so much from his (lower division) Software Construction. He is
    extremely knowledgeable about the Unix system and systems computing in
    general (which is one of my many interests).
\end{example}

\begin{example}
    For Computer Science, I wanted to study under Professor Alexander Sherstov
    because of his unparalleled passion for teaching. His Theory of Computation
    course (traditionally one of the most challenging courses in the Computer
    Science curriculum) was made intuitive only because we were being taught by
    someone who truly loved what he did. His teaching philosophy has no doubt
    transformed generations of students.

    \begin{quote}
        ``Learning is a process; thank you for letting me be a part of
        yours. As you all graduate and move on with your careers, I know the
        future of computing is in good hands. All the best with the next step in
        your careers!''
        \vspace{-1em}
        \begin{flushright}
            -- Professor Alexander Sherstov
        \end{flushright}
        \vspace{-1em}
    \end{quote}

    The mission statement he gave in his last lecture was also very compelling,
    and gave insight into his character.
    \begin{quote}
        ``When I took this course as an undergraduate, I sat where you sit now.
        The first time I learned about undecidability, I was in shambles.
        Surely, there must exist a computational algorithm for a given problem!
        It was a long walk back to the apartment that day, and food didn't taste
        the same for a couple of days... \it{But}, I had an idea. Even if most
        of the interesting problems are undecidable, some are not. So, I need
        not solve the general case; only the cases that matter. That is what I
        have dedicated my life's work towards. As I conclude the final lecture
        of the quarter, I hope you see the beauty of computation as I do, and I
        hope you learned something about the theory of computation.''
        \vspace{-1em}
        \begin{flushright}
            -- Professor Alexander Sherstov
        \end{flushright}
    \end{quote}
\end{example}

\begin{example}
    For Mathematics, I wanted to study under Professor Alexander Merkurjev
    because he has made significant contributions to the field of Algebra. The
    Group Theory course I took under him piqued my interest in higher
    abstractions of Mathematics.
\end{example}


\section{General Advice for Education}
As mentioned in the \bf{\nameref{sec:overview}}, I cannot give you in-depth
advice, even for a Computer Science major. What I \it{can} do is give some
general advice about approaching education.

\subsection{Grades}
Classes may not always seem useful or be the most engaging, but that's okay! To make
the most of it, try to truly \it{learn} the material, even if they may not seem
particularly relevant at the moment. While grades are important, it isn't as
important as actually knowing the material. Quoting my professor,
\begin{quote}
    ``If I did not have to give out grades, I would not. Only \it{you} know how
    well you have done in this course. Your grade is merely a reflection of
    three 2 hour \it{slivers} of time during the exams. I nor the teaching
    assitants will be able to tell you how much you know about the theory of
    computation. Your grade is just a letter; receiving an A and knowing nothing
    pales in comparison to receiving a C but taking the time to truly learn the
    course material. Be proud; this course is the most difficult course in the
    Computer Science curriculum here at UCLA. No matter how much you \it{think}
    you have learned, the reality of it is that you have learned so much more.''
    \vspace{-1em}
    \begin{flushright}
        -- Professor Alexander Sherstov
    \end{flushright}
\end{quote}

\subsection{Professors}
Talk to your professors! It sounds trivial but you would be surprised how empty
some office hours can be. Professors are experts in their field, and are a great
resource and mentors. Who knows, maybe they will offer you opportunities!
\begin{example}
    I got to know my Programming Languages professor very well, and we now climb
    together! I even presented an advanced topic in data structures as a guest
    lecturer for his introductory Data Structures class.
\end{example}

\subsection{Courses}
Take classes that are interesting to \it{you}. Explore outside of your major and
see what piques your interest (even if it doesn't \it{directly} help with your
major)!
\begin{example}
    I took proof-based math courses because math has always been interesting
    to me, and it has helped me tremendously in more ways than one. The
    learning curve was pretty steep, especially because we had to prove a lot of
    things that seemed obvious (e.g. ``$a = -(-a)$'' seems so obvious that
    it doesn't require a proof, but it does).
\end{example}

\subsection{Course Load}
Take a managable course load! ``Managable'' means something different to
everybody, so use your best judgement.
\begin{example}
    Throughout community college and the first year and a half at UCLA, I was
    working an average of 50 hours a week and taking four or more major-related
    classes a semester/quarter. It was not the best for my sleep schedule and I
    wasn't able to focus as well in class. I don't recommend it! Once I
    decreased my course load to three classes, I was able to better understand
    the material and get to know my professors better.
\end{example}


\newpage
\section{Statistics}
For those interested, these were my high school and community college
statistics.
\subsection{High School}
\begin{table}[ht]
    \centering
    \caption*{\it{Grade Point Averages}}
    \vspace{-0.8em}
    \begin{tabular}{l|c|c}
        Type           & Weighted & Non-weighted \\
        \hline
        Academic 9-12  & 4.1538   & 3.7179 \\
        Academic 10-12 & 4.3793   & 3.7931 \\
    \end{tabular}
    \vspace{-1em}
\end{table}

% freshman
\begin{table}[H]
    \centering
    \caption*{\it{Freshman Year}}
    \vspace{-0.8em}
    \begin{tabular}{l|l}
        Course          & Grade \\
        \hline
        English 1A      & A$-$  \\
        Latin 1A        & A$-$  \\
        Math IIA        & A     \\
        Biology A       & B     \\
        Cultl Gbl Age A & B     \\
    \end{tabular}
    \hspace{2em}
    \begin{tabular}{l|l}
        Course          & Grade \\
        \hline
        English 1B      & A     \\
        Latin 1B        & B$+$  \\
        Math IIA        & A     \\
        Biology B       & B$+$  \\
        Cultl Gbl Age B & B$+$  \\
    \end{tabular}
    \vspace{-1em}
\end{table}

% sophomore
\begin{table}[H]
    \centering
    \caption*{\it{Sophomore Year}}
    \vspace{-0.8em}
    \begin{tabular}{l|l}
        Course          & Grade \\
        \hline
        English 2A      & A     \\
        Latin 2A        & A     \\
        Math IIIA       & A$+$  \\
        Chemistry A     & A     \\
        World History A & A     \\
    \end{tabular}
    \hspace{2em}
    \begin{tabular}{l|l}
        Course          & Grade \\
        \hline
        English 2B      & A     \\
        Latin 2B        & A$-$  \\
        Math IIIB       & A$+$  \\
        Chemistry B     & A     \\
        World History B & A     \\
    \end{tabular}
    \vspace{-1em}
\end{table}

% junior
\begin{table}[H]
    \centering
    \caption*{\it{Junior Year}}
    \vspace{-0.8em}
    \begin{tabular}{l|l}
        Course           & Grade \\
        \hline
        H American Lit A & A$-$ \\
        H Latin 3A       & A$-$ \\
        H Precalculus A  & A$-$ \\
        AP Statistics A  & A    \\
        AP Physics 1A    & C    \\
        AP US History A  & B    \\
    \end{tabular}
    \hspace{2em}
    \begin{tabular}{l|l}
        Course           & Grade \\
        \hline
        H American Lit B & A     \\
        H Latin 3B       & A$-$  \\
        H Precalculus B  & A     \\
        AP Statistics B  & A     \\
        AP Physics 1B    & A$-$  \\
        AP US History B  & A$-$  \\
    \end{tabular}
    \vspace{-1em}
\end{table}

% senior
\begin{table}[H]
    \centering
    \caption*{\it{Senior Year}}
    \vspace{-0.8em}
    \begin{tabular}{l|l}
        Course           & Grade \\
        \hline
        AP Eng Lit A     & B     \\
        AP Latin A       & A$-$  \\
        AP Calc BC A     & B$-$  \\
        AP Com Sci A     & B$+$  \\
        AP Macro Econ    & A$-$  \\
        Beg Ceramics A   & A$+$  \\
    \end{tabular}
    \hspace{2em}
    \begin{tabular}{l|l}
        Course           & Grade \\
        \hline
        AP Eng Lit B     & CR    \\
        AP Latin B       & CR    \\
        AP Calc BC B     & CR    \\
        AP Com Sci B     & CR    \\
        AP Macro Econ    & CR    \\
        Beg Ceramics A   & CR    \\
    \end{tabular}
\end{table}
\newpage

\subsection{College}
\begin{table}[H]
    \centering
    \caption*{\it{Grade Point Average}}
    \vspace{-0.8em}
    \begin{tabular}{l|c}
        Type       & GPA  \\
        \hline
        Cumulative & 4.00 \\
        Department & 4.00 \\
    \end{tabular}
    \vspace{-1em}
\end{table}
% first year
\begin{table}[H]
    \centering
    \caption*{\it{First Year}}
    \vspace{-0.8em}
    \begin{tabular}{l|l}
        Course                                 & Grade \\
        \hline
        C Programming                          & A \\
        H Psychology 1                         & A \\
        College Writing 2                      & A \\
        Analytical Geometry/Calculus III       & A \\
    \end{tabular}
    \hfill
    \begin{tabular}{l|l}
        Course                                 & Grade \\
        \hline
        H Communications                       & A \\
        Java Programming                       & A \\
        Physics I (Kinematics)                 & A \\
        Intro. to Linear Algebra               & A \\
        Intro. to Computer Systems             & A \\
    \end{tabular}
    \vspace{-1em}
\end{table}

% summer session
\begin{table}[H]
    \centering
    \vspace{-0.8em}
    \caption*{\it{Summer Session}}
    \begin{tabular}{l|l}
        Course                                 & Grade \\
        \hline
        Intro. to Computer Science I           & A \\
        H Principles of Microeconomics         & A \\
        Physics II (Electricity \& Magnetism)  & A \\
    \end{tabular}
    \vspace{-1em}
\end{table}

% second year
\begin{table}[H]
    \centering
    \vspace{-0.8em}
    \caption*{\it{Second Year}}
    \begin{tabular}{l|l}
        Course                                 & Grade \\
        \hline
        Assembly Language I                    & A \\
        H Film \& US Culture                   & A \\
        Discrete Mathematics I                 & A \\
        Intro. to Computer Science II          & A \\
        Elementary Differential Equations      & A \\


    \end{tabular}
    \hfill
    \begin{tabular}{l|l}
        Course                                 & Grade \\
        \hline
        Assembly Language II                   & A \\
        Discrete Mathematics II                & A \\
        H Academic, Career, Life               & A \\
        Intro. to Computer Science III         & A \\
        Data Structures and Algorithms         & A \\
    \end{tabular}
\end{table}




things i would like to talk about
\begin{itemize}[label=$\to$]
    \item CSU or UC?
    \item School name
    \item Major choice
    \item Outside of CA
\end{itemize}
\end{document}


\end{document}

% PAGE BREAK COMMENT
% \begin{center}
%     \vspace{5em}
%     \bf{The rest of this page is intentionally left blank.}
% \end{center}
%
% \newpage

